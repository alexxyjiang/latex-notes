\item{\textbf{逆矩阵与伪逆矩阵的异同}}\label{q1.1}

\begin{enumerate}
\item{行列式及伴随矩阵的计算}

给定方阵$\mathbf{A}$,可以根据其行列式$\mathit{det}(\mathbf{A})$,定义其余子式矩阵$\mathbf{C}$,满足
\begin{equation*}
c_{ij} = (-1)^{i+j} * \mathit{det}\begin{bmatrix}
a_{1,1} & \dots & a_{1,j-1} & \Box & a_{1,j+1} & \dots & a_{1,n} \\
\vdots & \vdots & \vdots & \Box & \vdots & \vdots & \vdots \\
a_{i-1,1} & \dots  & a_{i-1,j-1} & \Box & a_{i-1,j+1} & \dots  & a_{i-1,n} \\
\Box & \Box & \Box & \Box & \Box & \Box & \Box \\
a_{i+1,1} & \dots  & a_{i+1,j-1} & \Box & a_{i+1,j+1} & \dots  & a_{i+1,n} \\
\vdots & \vdots & \vdots & \Box & \vdots & \vdots & \vdots \\
a_{m,1} & \dots & a_{m,j-1} & \Box & a_{m,j+1} & \dots & a_{m,n}
\end{bmatrix}
\end{equation*}
A的伴随矩阵$\mathit{adj}(\mathbf{A}) = \mathbf{C}^{T}$
当$\mathit{det}(\mathbf{A})$有定义且$\neq 0$(非奇异方阵)时,A的逆矩阵$A^{-1} = \frac{\mathit{adj}(\mathbf{A})}{\mathit{det}(\mathbf{A})}$。

\item{伪逆矩阵定义及计算}

X

\end{enumerate}
