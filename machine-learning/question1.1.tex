\item{\textbf{逆矩阵与伪逆矩阵}}\label{q1.1}

\begin{enumerate}
\item{逆矩阵}

给定方阵$\mathbf{A}$,可以根据其行列式$\mathit{det}(\mathbf{A})$,定义其余子式矩阵$\mathbf{C}$,满足
$$c_{ij} = (-1)^{i+j} * \mathit{det}\begin{bmatrix}
a_{1,1} & \dots & a_{1,j-1} & \Box & a_{1,j+1} & \dots & a_{1,n} \\
\vdots & \vdots & \vdots & \Box & \vdots & \vdots & \vdots \\
a_{i-1,1} & \dots  & a_{i-1,j-1} & \Box & a_{i-1,j+1} & \dots  & a_{i-1,n} \\
\Box & \Box & \Box & \Box & \Box & \Box & \Box \\
a_{i+1,1} & \dots  & a_{i+1,j-1} & \Box & a_{i+1,j+1} & \dots  & a_{i+1,n} \\
\vdots & \vdots & \vdots & \Box & \vdots & \vdots & \vdots \\
a_{m,1} & \dots & a_{m,j-1} & \Box & a_{m,j+1} & \dots & a_{m,n}
\end{bmatrix}$$
A的伴随矩阵$\mathit{adj}(\mathbf{A}) = \mathbf{C}^{T}$
当$\mathit{det}(\mathbf{A})$有定义且$\neq 0$(非奇异方阵)时,A的逆矩阵
$$A^{-1} = \frac{\mathit{adj}(\mathbf{A})}{\mathit{det}(\mathbf{A})}$$

\item{伪逆矩阵}

当矩阵$\mathbf{A}$不是方阵,或$\mathbf{A}$是奇异方阵时,无法直接获得其逆矩阵。此时,若$\mathbf{AGA} = \mathbf{A}$,则称矩阵$\mathbf{G}$为矩阵$\mathbf{A}$的伪逆矩阵。显然,当矩阵$\mathbf{A}$可逆时,其逆矩阵和伪逆矩阵相同。若
$$\mathbf{A} = \mathbf{P} \begin{bmatrix}\mathbf{I}_{r} & 0 \\ 0 & 0 \end{bmatrix} \mathbf{Q}$$
其中$\mathbf{P}$及$\mathbf{Q}$为可逆矩阵,则伪逆矩阵
$$\mathbf{G} = \mathbf{Q}^{-1} \begin{bmatrix} \mathbf{I}_{r} & \mathbf{U} \\ \mathbf{W} & \mathbf{V} \end{bmatrix} \mathbf{P}^{-1}$$
其中$\mathbf{U}, \mathbf{V}, \mathbf{W}$为任意矩阵。

\end{enumerate}
